\chapter*{CAPÍTULO II \\ Dos princípios e objetivos}


\textbf{ARTIGO 2º} - São os princípios do AREA31:

\begin{enumerate}[label=\Roman* -]
    \item O acesso livre e universal ao conhecimento gerado 
          sob suas premissas;
    \item O financiamento de suas atividades será, majoritariamente, 
          promovido por seus próprios associados;
    \item Garantir a livre iniciativa aos associados no que concerne a 
          organização, promoção e desenvolvimento de atividades e 
          projetos, sejam de natureza individual ou coletiva.
\end{enumerate}


\textbf{ARTIGO 3º} - São os objetivos do AREA31:

\begin{enumerate}[label=\Roman* -]
    \item Fomentar o desenvolvimento de uma comunidade de interessados em 
          inovação, ciência, cultura, tecnologia, criatividade, artes 
          e disseminação do conhecimento;
    \item Promover os ideais da ética hacker perante a sociedade e o 
          Estado, sempre na busca de esclarecer as finalidades 
          políticas-sociais desta comunidade;
    \item Promover e apoiar o uso de tecnologias e padrões que permitam 
          seu livre uso, estudo, adaptação, modificação e compartilhamento, 
          respeitando a autonomia individual e coletiva, bem como 
          incentivando a colaboração compartilhada;
    \item Promover o acesso à tecnologia e a informação como um Direito 
           Constitucional de todo cidadão;
    \item Incentivar o livro acesso à educação, à cultura e ao conhecimento;
    \item Promover o desenvolvimento econômico e social sustentável, a 
          ética, a paz, a cidadania, aos direitos humanos, a democracia e o 
          combate à pobreza.
\end{enumerate}


\textbf{ARTIGO 4º} - Para consecução de seus objetivos o AREA31 utilizar-se-á, 
dentre outros, dos seguintes instrumentos:

\begin{enumerate}[label=\Roman* -]
    \item prover infraestrutura física e lógica, espaço, equipamentos, 
        ferramentas, materiais e serviços para a realização, 
        por livre iniciativa individual ou em grupo, de projetos educacionais,
        técnico-científicos e artísticos;
    \item manter espaços de convivência seguros, convidativos, amigáveis 
        e adequados para que seus Associados e o público em geral possam 
        realizar interações sociais compatíveis com seus objetivos;
    \item realizar atividades de disseminação do conhecimento 
        técnico-científico e artístico na forma de estudos, análises, 
        eventos, reuniões, exposições, oficinas, produções audiovisuais, 
        páginas eletrônicas, material informativo e publicações para seus 
        Associados e para o público em geral;
    \item organizar eventos culturais, sociais, artísticos e recreativos 
        com o objetivo de promover a socialização entre seus Associados 
        e deles com o público em geral; 
    \item relacionar-se com entidades congêneres, nacionais ou 
        estrangeiras, visando desenvolver intercâmbio institucional.
\end{enumerate}


\textbf{Parágrafo único} - No cumprimento de seus objetivos, o AREA31 poderá 
firmar contratos e/ou convênios com entidades financiadoras de projetos, 
nacionais ou estrangeiras, de direito público ou privado, que tenham 
princípios similares ou complementares aos seus, destinando os recursos 
exclusivamente para a manutenção e desenvolvimento de seus objetivos, 
sempre em conformidade com a legislação em vigor.
\\

\textbf{ARTIGO 5º} - No desenvolvimento de suas atividades, o AREA31 
observará os princípios da legalidade, impessoalidade, moralidade, 
publicidade, economicidade e eficiência, não realizando qualquer 
discriminação de raça, sexo, orientação sexual, nacionalidade, 
credo religioso, convicções políticas e condição social, intelectual ou 
seus contrários.
A participação de menores de 18 anos nas atividades do AREA31, quando cabível,
será permitida mediante autorização ou acompanhamento de responsável legal.
A participação nas atividades do AREA31 será vetada apenas àqueles que, 
por descumprimento deste Estatuto ou do Código de Conduta, tenham sido 
expulsos do AREA31 ou estejam com seus direitos estatutários suspensos.

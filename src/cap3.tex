\chapter*{CAPÍTULO III \\ Dos associados}


\textbf{ARTIGO 6º} - O AREA31 contará com um número ilimitado de Associados, 
podendo filiar-se somente pessoas físicas maiores de 18 (dezoito) anos, 
distintos em três categorias:

\begin{itemize}
    \item \emph{Associado Fundador}: aquele que tenha participado da 
        Assembleia de Fundação do AREA31 ou que tenha se associado a ele 
        até 30 (trinta) dias após a sua fundação, tendo realizado o aporte 
        patrimonial determinado nesta Assembleia.
    \item \emph{Associado Titular}: pessoa física que tenha sua proposta de 
        associação aprovada por instância competente;
    \item \emph{Associado Honorário}: título simbólico concedido a pessoa de 
        notório saber, que tenha feito contribuições de reconhecido valor ao 
        campo do conhecimento, ou que tenha contribuído, moral ou 
        materialmente, de maneira significativa, para o 
        engrandecimento do AREA31.
\end{itemize}

\textbf{Parágrafo 1º} - Será designado genericamente por Associado Efetivo 
todo aquele que pertença às categorias de Associado Fundador ou Associado 
Titular e que esteja em pleno gozo de seus direitos estatutários.

\bigskip

\textbf{Parágrafo 2º} - O Título de Associado Honorário, quando conferido a 
Associado Efetivo, não o priva dos direitos nem o exime dos deveres 
inerentes a essa categoria.

\bigskip

\textbf{ARTIGO 8º} - São DIREITOS dos Associados:

\begin{itemize}
    \item gozar dos benefícios oferecidos pelo AREA31 na forma prevista 
        neste Estatuto Social e no Regimento Interno; 
    \item recorrer à Assembleia Geral contra qualquer ato da Diretoria 
        Executiva, do Conselho Deliberativo ou do Conselho Fiscal;
    \item estar presente e tendo a voz de intermedio através de um 
        representante do órgão administrativo do AREA31.
\end{itemize}


\textbf{ARTIGO 9º} - São DEVERES dos Associados:

\begin{itemize}
    \item cumprir e fazer cumprir o presente Estatuto Social, o Código de 
        Conduta e o Regimento Interno; 
    \item respeitar e cumprir as decisões da Assembleia Geral, do Conselho 
        Deliberativo e da Diretoria Executiva;
    \item zelar pelo bom nome do AREA31;
    \item defender o patrimônio e os interesses do AREA31;
\end{itemize}

\textbf{ARTIGO 10º} - São DIREITOS exclusivos dos Associados Efetivos:

\begin{itemize}
    \item votarem e serem votados em Assembleia Geral;
    \item candidatarem-se para os cargos da Diretoria Executiva e 
        Conselho Fiscal;
    \item requerer mandato no Conselho Deliberativo, conforme disposto 
        no artigo 23;
    \item livre acesso a todos os arquivos, documentos e instalações do AREA31.
\end{itemize}

\textbf{Parágrafo único} - O Associado Efetivo que pertença à categoria de 
Titular somente poderá candidatar-se e ocupar cargo na Diretoria Executiva 
após 12 (doze) meses transcorridos de sua admissão ao quadro social.

\bigskip

\textbf{ARTIGO 11º} - São DEVERES exclusivos dos Associados Efetivos:

\begin{itemize}
    \item comparecer por ocasião das Assembleias Gerais Ordinárias;
    \item honrar pontualmente com suas contribuições associativas;
    \item denunciar qualquer irregularidade verificada dentro do AREA31 ao 
        Conselho Deliberativo, para que a Assembleia Geral analise 
        e tome as providências cabíveis.
\end{itemize}

\textbf{Parágrafo único} - Serão considerados em pleno gozo de seus direitos 
estatutários apenas os Associados que estejam em cumprimento com o 
disposto nos incisos deste artigo.

\bigskip

\textbf{ARTIGO 12º} - A admissão dos Associados dar-se-á de forma 
independente de classe social, nacionalidade, sexo, raça, cor e crença 
religiosa - ou a falta destas - e, para seu ingresso, o interessado deverá 
submeter sua proposta de admissão para apreciação do Conselho Deliberativo, 
de acordo com os critérios definidos no Regimento Interno.

\bigskip

\textbf{Parágrafo 1º} - O título de Associado é pessoal e intransferível.

\bigskip

\textbf{Parágrafo 2º} - A associação está vinculada ao pagamento de 
contribuições associativas com valores e periodicidade a serem especificados 
no Regimento Interno.

\bigskip

\textbf{ARTIGO 13º} - É direito do Associado demitir-se quando julgar 
necessário, protocolando, junto ao Conselho Deliberativo, 
seu pedido de demissão.

\bigskip

\textbf{ARTIGO 14º} - A perda da qualidade de Associado dar-se-á por 
falecimento, incapacidade ou justa causa, sendo esta última cabível 
nas seguintes hipóteses:

\begin{itemize}
    \item descumprimento deste Estatuto Social, do Código de Conduta ou do 
        Regimento Interno;
    \item prática de ato ilícito e/ou incompatível com os princípios do AREA31;
    \item difamação do AREA31 ou de seus Associados;
    \item prática de ato que contrarie decisões de Assembleias, 
        Diretoria e Conselhos;
    \item não pagamento de três parcelas consecutivas das contribuições 
        associativas, podendo ter o retorno assim que quitar as dívidas. 
    \item Citar-se como membro do area 31 para promoção de serviços, 
        historico curricular ou qualquer monetização ou otimização de 
        processos ou adquirir facilidades de modo não direcionados aos 
        indicados pelo conselho formado em reuniao na Area31, 
        com no minimo 10 presentes.
\end{itemize}

\textbf{Parágrafo único} - A perda da qualidade de Associado por justa causa 
será determinada pelo Conselho Deliberativo, cabendo recurso da decisão à 
Assembleia Geral, unicamente convocada para este fim, sendo garantida a 
ampla defesa em todas as instâncias.

\bigskip

\textbf{ARTIGO 15º} - Os associados não respondem, nem  mesmo solidária ou 
subsidiariamente, pelos encargos e obrigações sociais do AREA31.

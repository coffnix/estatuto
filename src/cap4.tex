\chapter*{CAPÍTULO IV \\ Da Assembléia geral}

\textbf{ARTIGO 16º} - A Assembleia Geral é o órgão soberano do AREA31, 
composto pelos Associados Efetivos reunidos para deliberar sobre matérias 
de interesse da associação.

\bigskip

\textbf{Parágrafo único} - São órgãos do AREA31, independentes e harmônicos 
entre si, o Conselho Deliberativo, o Conselho Fiscal e a Diretoria 
Executiva, estando eles diretamente subordinados à Assembleia Geral.

\bigskip

\textbf{ARTIGO 17º} - Compete privativamente à Assembleia Geral:

\begin{itemize}
    \item eleger a Diretoria Executiva e o Conselho Fiscal; 
    \item deliberar sobre a compra ou alienação de bens imóveis;
    \item deliberar sobre a celebração de convênios, aquisições ou contratos 
        com valor superior a 20 salários mínimos;
    \item destituir a Diretoria Executiva e o Conselho Fiscal, inteiramente 
        ou seus membros, e o Presidente do Conselho Deliberativo; 
    \item alterar o Estatuto Social;
    \item deliberar quanto à dissolução da Associação;
    \item decidir em última instância.
\end{itemize}


\textbf{ARTIGO 18º} - Exceto pelo disposto no Artigo 19º, a Assembleia Geral 
instalar-se-á em primeira convocação com a presença da maioria absoluta dos 
Associados Efetivos e, nas convocações seguintes, no mínimo meia hora e no 
máximo 24 horas após a primeira, com pelo menos 07 (sete) Associados Efetivos.

\bigskip

\textbf{Parágrafo único} - As deliberações da Assembleia Geral serão tomadas 
por maioria absoluta dos votos dos presentes.

\bigskip

\textbf{ARTIGO 19º} - As deliberações a que se referem os incisos 4, 5 e 6 
do Artigo 17º, são de competência exclusiva de Assembleia Geral 
Extraordinária unicamente convocada para esse fim, não podendo ela 
deliberar, em primeira convocação, sem a maioria absoluta dos Associados 
Efetivos, ou com menos de 1/3 (um terço) deles nas convocações seguintes.  

\bigskip

\textbf{Parágrafo único} - além do especificado no caput deste artigo, para 
deliberar sobre a reforma do Estatuto Social ou a dissolução da Associação 
(incisos 5 e 6, respectivamente), é necessária a presença de todos os 
Associados Fundadores constantes do quadro social, em primeira convocação, 
ou ao menos a maioria absoluta deles nas convocações seguintes.

\bigskip

\textbf{ARTIGO 20º} - Exceto para as deliberações a que se refere o inciso 1 
do Artigo 17º, quando será admitido o escrutínio secreto, as votações da 
Assembleia Geral deverão ser realizadas na forma de escrutínio público, 
sendo permitido o uso de meios eletrônicos, quando julgar-se necessário, 
na forma especificada no Regimento Interno.

\bigskip

\textbf{ARTIGO 21º} - Assembleia Geral reunir-se-á ordinariamente uma vez 
ao ano, até o dia 31 de março, e extraordinariamente, quando convocada pelo 
Presidente da Diretoria Executiva, pelo Conselho Deliberativo, pelo Conselho 
Fiscal, ou ainda ao menos 1/3 (um terço) dos Associados Efetivos, que 
subscreverão e especificarão os motivos da convocação.

\bigskip

\textbf{Parágrafo 1º} - A Assembleia Geral Ordinária ocorrerá anualmente e 
deverá deliberar, quando cabível, sobre a eleição da Diretoria Executiva e 
do Conselho Fiscal, além de aprovar a prestação de contas do exercício social 
findo e a previsão orçamentária do exercício entrante.

\bigskip

\textbf{Parágrafo 2º} - A Assembleia Geral será convocada mediante edital 
afixado na sede do AREA31, além de meios eletrônicos, conforme detalhado 
no Regimento Interno, contendo data, horário, local e a ordem do dia, 
com antecedência mínima de 05 (cinco) dias úteis.

\bigskip

\textbf{Parágrafo 3º} - A Assembleia Geral será presidida por um Associado 
Efetivo indicado pelos presentes, que comporá a mesa com o Secretário, 
a quem cumprirá elaborar a ata dos trabalhos.

\bigskip

\textbf{Parágrafo 4º} - No caso de deliberação referente ao disposto no 
inciso 1 do Artigo 17, o Associado indicado para presidir a mesa não poderá 
ser um dos concorrentes ao pleito.

\bigskip

\textbf{Parágrafo 5º} - A Assembleia Geral poderá deliberar e autorizar que a 
ata dos trabalhos seja assinada somente pelos integrantes da mesa, 
desde que assinada a lista de presença pelos Associados presentes.

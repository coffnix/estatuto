\chapter*{CAPÍTULO VII \\ Da diretoria executiva}


\textbf{ARTIGO 33º} - A Diretoria Executiva é o órgão responsável pela 
administração do AREA31 e pela implementação da política 
estabelecida pelo Conselho Deliberativo.

\bigskip

\textbf{ARTIGO 34º} - A Diretoria Executiva será composta por 05 (cinco) 
membros, assim discriminados: Presidente, Vice-Presidente, Diretoria de 
Tecnologia, Diretoria de Comunicação, Diretoria Financeira e reunir-se-á 
sempre que houver convocação do Presidente ou de quaisquer 
02 (dois) de seus membros.

\bigskip

\textbf{Parágrafo 1º} - Os membros da Diretoria Executiva serão eleitos em 
Assembleia Geral Ordinária, conforme o disposto nos Artigos 41 
e 43 deste Estatuto. 

\bigskip

\textbf{Parágrafo 2º} - Os mandatos da Diretoria Executiva terão duração de 
02 (dois) anos, sendo vedada a recondução de um Diretor para o mesmo cargo.

\bigskip

\textbf{ARTIGO 35º} - Compete à Diretoria Executiva:

\begin{itemize}
    \item administrar o AREA31 e seu patrimônio de acordo com o presente 
        Estatuto e implementando as deliberações da Assembleia Geral e do 
        Conselho Deliberativo, promovendo o bem geral da 
        entidade e dos Associados;
    \item cumprir e fazer cumprir o presente Estatuto;
    \item instituir e destituir comissões ou grupos de trabalho com a 
        função de auxiliá-la em suas funções;
    \item representar e defender os interesses dos Associados;
    \item zelar pelo cumprimento do orçamento anual, e pela lisura das 
        operações e demonstrações financeiras;
    \item apresentar à Assembleia Geral Ordinária o relatório de sua 
        gestão, e prestar contas referentes ao exercício financeiro;
    \item comparecer às reuniões do Conselho Deliberativo, 
        de modo a promover a sinergia entre os dois órgãos;
    \item manter em contas bancárias os valores do AREA31, devendo aplicá-lo 
        de acordo com as deliberações do Conselho Deliberativo 
        ou da Assembleia Geral.
\end{itemize}

\textbf{Parágrafo 1º} - As decisões da Diretoria Executiva deverão ser tomadas 
por maioria simples dos votos dos presentes, com participação necessária 
do Presidente e de no mínimo mais dois de seus membros, 
cabendo ao Presidente, em caso de empate, o Voto de Minerva.

\bigskip

\textbf{Parágrafo 2º} - Para a emissão de cheques ou realização de transações 
bancárias serão necessárias, respectivamente, as assinaturas ou autorização 
de dois diretores, sendo que um deles deve ser, obrigatoriamente, 
o  Diretor Financeiro e Presidente.

\bigskip

\textbf{ARTIGO 36º} - Compete ao Presidente:

\begin{itemize}
    \item representar o AREA31 ativa e passivamente, perante os Órgãos 
        Públicos, Judiciais e Extrajudiciais, em juízo ou fora dele, 
        podendo delegar poderes e constituir advogados ou procuradores 
        para o fim que julgar necessário;
    \item convocar e presidir as reuniões da Diretoria Executiva;
    \item convocar a Assembleia Geral, conforme o disposto no Artigo 21;
    \item organizar um relatório contendo balanço do exercício financeiro e 
        os principais eventos do ano anterior, apresentando-o anualmente à 
        Assembleia Geral Ordinária e trimestralmente ao Conselho Deliberativo;
    \item representar o AREA31 perante outras entidades públicas ou privadas 
        externas, quando requisitado por estas, ou ainda em eventos, 
        premiações e comemorações oficiais;
    \item substituir o Secretário em suas eventuais faltas e impedimentos.
\end{itemize}

\textbf{ARTIGO 37º} - Compete ao Secretário:

\begin{itemize}
    \item redigir e manter em dia as atas das Assembleias Gerais e das 
        reuniões do Conselho Deliberativo e da Diretoria Executiva;
    \item manter e ter sob guarda os arquivos do AREA31, incluindo documentos, 
        projetos, relatórios e códigos-fonte produzidos internamente;
    \item dirigir e supervisionar todo o trabalho de secretaria;
    \item zelar pelo bom funcionamento dos sistemas de informação do AREA31;
    \item substituir os Diretores de Hardware e Software em suas eventuais 
        faltas e impedimentos.
\end{itemize}

\textbf{ARTIGO 38º} - Compete ao Tesoureiro:

\begin{itemize}
    \item abrir e movimentar contas bancárias, assinar cheques ou documentos 
        contábeis, executar ordens de pagamento, recebimentos e 
        transferências bancárias;
    \item administrar o recebimento de contribuições associativas, 
        tomando as medidas cabíveis quando do seu não cumprimento no 
        prazo e forma estipulados pelo Regimento Interno;
    \item supervisionar o trabalho de assessorias de tesouraria ou 
        contabilidade que venham a ser contratadas;
    \item apresentar ao Conselho Fiscal, anualmente e sempre que solicitado, 
        balancetes fiscais e financeiros; 
    \item substituir o Presidente em suas eventuais faltas e impedimentos.
\end{itemize}

\textbf{ARTIGO 39º} - Compete ao Diretor de Hardware:

\begin{itemize}
    \item registrar aquisições e doações ao AREA31;
    \item zelar pela conservação do patrimônio e da infraestrutura do AREA31;
    \item zelar pela sede do AREA31, controlando o acesso a ela na forma 
        estabelecida pelo Regimento Interno;
    \item confeccionar e manter a relação dos bens do AREA31, deixando-a 
        disponível à consulta dos Associados e apresentando-a, 
        quando solicitada, aos demais órgãos do AREA31;
    \item substituir o Tesoureiro em suas eventuais faltas e impedimentos.
\end{itemize}

\textbf{ARTIGO 40º} - Compete ao Diretor de Software:

\begin{itemize}
    \item manter atualizado o quadro social;
    \item fomentar a sociabilização entre os Associados;
    \item propor ou coordenar a realização de eventos técnicos e sociais;
    \item promover a comunicação interna do AREA31 e incentivar a troca de 
        informações entre os Associados;
    \item promover ações visando o aumento do quadro social;
    \item promover a divulgação externa do AREA31, tornando de conhecimento 
        público os valores e as atividades realizadas por ele.
\end{itemize}

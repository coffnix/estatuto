\chapter*{CAPÍTULO V \\ Do conselho deliberativo}


\textbf{ARTIGO 22º} - O Conselho Deliberativo é o órgão deliberativo 
responsável pela política a ser observada pelo AREA31, tanto na consecução 
de seus objetivos sociais como no planejamento financeiro e no desenvolvimento 
das relações do AREA31 com o corpo social, com a sociedade civil e com 
pessoas físicas ou jurídicas com as quais mantenha ou venha a manter 
vínculos de qualquer natureza e será composto por no mínimo 7 (sete) e 
no máximo 15 (quinze) Associados Efetivos.

\bigskip

\textbf{Parágrafo 1º} - Os membros do Conselho Deliberativo serão designados 
por "Deliberativos".

\bigskip

\textbf{Parágrafo 2º} - O Conselho Deliberativo será presidido pelo 
Presidente do conselho do AREA31.

\bigskip

\textbf{ARTIGO 23º} - Poderá ocupar o cargo de Deliberativo qualquer 
Associado Efetivo, mediante requisição por escrito ao Presidente do 
conselho, que deverá ser automaticamente aceita e registrada.

\bigskip

\textbf{Parágrafo 1º} - Terá seu mandato automaticamente revogado o 
Deliberativo que ausentar-se de 03 (três) ou mais reuniões sem justificativa 
aceita pelo plenário. As ausências não justificadas serão abonadas anualmente, 
na data de realização da Assembleia Geral Ordinária.

\bigskip

\textbf{Parágrafo 2º} - Para efeitos de computação de quorum, não serão 
considerados os Deliberativos com mandato revogado ou com justificativa de 
ausência aceita pelo plenário.

\bigskip

\textbf{Parágrafo 3º} - O Deliberativo cujo mandato seja revogado por não 
comparecimento poderá requerer o reingresso ao plenário passadas ao menos 
03 (três) reuniões ordinárias da data de revogação.

\bigskip

\textbf{ARTIGO 24º} - Compete ao Conselho Deliberativo:

\begin{itemize}
    \item estabelecer as diretrizes básicas e planos de ação do AREA31;
    \item zelar pela observância do Estatuto e cumprir as decisões da 
        Assembleia Geral;
    \item elaborar, aprovar e alterar o Regimento Interno;
    \item elaborar e aprovar o orçamento do exercício social entrante, 
        e apresentá-lo à Assembleia Geral;
    \item aprovar e apresentar à Assembleia Geral o relatório e a prestação 
        de contas do exercício findo, com parecer do Conselho Fiscal;
    \item decidir sobre a participação institucional do AREA31 
        em eventos externos;
    \item indicar e destituir, a qualquer momento, o Presidente do 
        conselho do AREA31;
    \item manifestar-se, através do Presidente do conselho, por qualquer meio 
        de comunicação, em nome do AREA31, sobre assunto de interesse público;
    \item proceder à indicação de membro interino do Conselho Fiscal ou da 
        Diretoria Executiva, quando houver vacância em algum destes cargos;
    \item decidir sobre a admissão e exclusão de Associados, conforme o 
        disposto neste Estatuto e no Regimento Interno;
    \item decidir sobre o valor das contribuições associativas;
    \item conceder, mediante a devida fundamentação, isenção de contribuição 
        associativa para Associado Efetivo em situação de extrema necessidade;
    \item conceder o título de Associado Honorário;
    \item decidir sobre a celebração de compras, contratos e convênios de 
        qualquer espécie e aluguel de imóveis com valor inferior a 
        20 salários mínimos;
    \item decidir sobre o recebimento de doações de pessoas físicas ou 
        jurídicas externas ao quadro social;
    \item encaminhar proposta de reforma do Estatuto à Assembleia Geral;
    \item convocar a Assembleia Geral; e
    \item decidir sobre casos omissos deste Estatuto Social.
\end{itemize}

\textbf{ARTIGO 25º} - O Conselho Deliberativo reunir-se-á ordinariamente, 
com periodicidade estabelecida no Regimento Interno, ou extraordinariamente, 
por solicitação do Presidente do conselho, da Diretoria Executiva ou por 
requerimento subscrito por, no mínimo, dois terço (2/3) 
dos Associados Efetivos.

\bigskip

\textbf{Parágrafo 1º} - Para sua instalação, a reunião do Conselho 
Deliberativo deverá contar com a presença mínima de 2/3 
(dois terço) dos Deliberativos.

\bigskip

\textbf{Parágrafo 2º} - As reuniões ordinárias do conselho deverão ocorrer, 
no mínimo, uma vez a cada dois meses e não mais que uma vez por semana. 

\bigskip

\textbf{Parágrafo 3º} - As reuniões extraordinárias do conselho deverão ser 
convocadas com no mínimo 02 (dois) dias úteis de antecedência, 
na forma estabelecida pelo Regimento Interno.

\bigskip

\textbf{ARTIGO 26º} - O Presidente do conselho será eleito por maioria 
simples dos votos dos Deliberativos, com mandato de 02 (dois) ano, 
sendo sempre permitida a recondução.

\textbf{Parágrafo único} - O Presidente do conselho não poderá ocupar cargo 
na Diretoria Executiva.

\bigskip

\textbf{Artigo 27º} - Compete ao Presidente do conselho:

\begin{itemize}
    \item presidir as reuniões do Conselho Deliberativo;
    \item indicar um Deliberativo que o substitua em sua ausência;
    \item receber e registrar requisições de ingresso no Conselho Deliberativo;
    \item realizar o controle de presença dos membros do Conselho Deliberativo;
    \item representar o Conselho perante outras instâncias deliberativas e administrativas do AREA31;
    \item servir de porta-voz preferencial do AREA31 perante a imprensa e a comunidade externa; e
    \item cumprir e fazer cumprir as decisões do Conselho Deliberativo.
\end{itemize}

\textbf{Parágrafo único} - na ausência do Presidente do conselho e de seu 
substituto, o Conselho Deliberativo deve ser presidido interinamente por um 
Deliberativo eleito pela maioria simples dos votos dos presentes, 
desde que este não ocupe cargo na Diretoria Executiva.

\bigskip

\textbf{ARTIGO 28º} - As decisões do Conselho Deliberativo serão tomadas por 
maioria simples, cabendo apenas um voto a cada membro presente. 

\bigskip

\textbf{Parágrafo único} - Em caso de empate na votação, a matéria deverá 
ser postergada até a próxima reunião ordinária. 
Persistindo o empate, o Presidente do conselho terá direito ao Voto de Minerva.
